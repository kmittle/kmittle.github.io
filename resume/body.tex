\logosection{\faGraduationCap}{教育经历}
\datedline{\textbf{复旦大学}}
{\dateRange{2021.9}{至今}}
{\small
类脑智能算法与应用 \quad 直博生 \quad 类脑智能科学与技术研究院
\begin{itemize}
\item GPA:3.50/4.00
\item 主要从事\textbf{以人为中心的AIGC}、及\textbf{AI与神经科学}交叉领域等研究;
\item 师从类脑研究院单洪明研究员、及华山医院/工研院双聘教授李郁欣教授进行指导;
\item 曾前往\textbf{香港中文大学},师从\textbf{袁奕萱}教授课题组进行交流学习;
  \item 现一作发表2篇\textbf{AI交叉顶刊SCI Q1},1篇\textbf{CCF A顶会},及4篇一作论文在审,参与其他论文11篇,5项\textbf{公开发明专利};
\end{itemize}
}
\datedline{\textbf{武汉理工大学}}
{\dateRange{2016.9}{2021.6}}
{\small
软件工程专业\quad 本科\quad 计算机与人工智能学院
\begin{itemize}
  \item GPA:4.41/5.0,计算机专业课程 GPA:4.50/5.00,综测: 1/326;
  \item 曾获评\textbf{武汉理工十大风云学子} (\textbf{1\%\%})、校优秀学生标兵等30余项荣誉,并作为学生代表在毕业典礼上发言;
  \item 曾\textbf{连续两年}获得\textbf{本科国家奖学金},并大四获得全校最高奖学金(\textbf{卓越奖学金, 3\%\%});
  \item 曾于武汉大学外国语学院法语专业进行辅修学习;
\end{itemize}
}
\logosection{\faSuitcase}{项目经历}
\datedline{\textbf{多模态时空感知的阿尔兹海默症智能诊断研究}}{\dateRange{2021.09}{至今}}
{\small
\begin{itemize}
  \item \textbf{项目简述}: 与\underline{\emph{腾讯犀牛角基金项目}}及\underline{\emph{复旦大学华山医院}}合作,本课题利用多种医学影像模态,如3D 结构磁共振成像(MRI)和病理正电子扫描(PET)成像,的互补特性,i)利用\emph{AdaIN}模块开发基于\emph{GAN}的双向共享合成网络;ii)及利用\emph{3D latent DiT}框架进行统一编码不同模态,并实现MRI到多示踪剂PET的跨模态合成; iii) 提出\underline{混合粒度排序原型学习}实现表征排序,从而利用大量横向疾病数据实现患者纵向状态预测。iv)基于\underline{Mixture-of-Experts}架构,将单模态MRI学生模型与多模态MRI和PET教师模型进行\underline{分层蒸馏对齐},提升单模态性能。
  \item \textbf{项目成果}:以第一作者身份发表于\underline{\emph{《Medical Image Analysis》}}及\underline{\emph{《Journal of Biomedical and Health Informatics》}}各一篇(均为SCI Q1),而另一篇也以第一作者身份在审于\underline{\emph{《Research》}},同时参与发表相关发明专利2项(已公开)。
\end{itemize}
}
\datedline{\textbf{治疗感知的脑胶质瘤世界模型}}
{\dateRange{2025.09}{至今}}
{\small
\begin{itemize}
  \item \textbf{项目简述}:在\underline{香港中文大学}交流学习期间,借助多治疗时间节点(手术、放射化疗、药物化疗等)的多模态脑胶质瘤数据,开发了一个3D脑胶质瘤(原发性脑癌)世界模型,基于\underline{统一多模态模型框架(UMM)}, 利用LoRA微调预训练show-o2实现治疗方案理解以及治疗后 MRI 预测。i)设计 Y-shaped \underline{混合transformer(MoT)}架构进行表征解耦和有机统一;ii)模型包含两条分支:用于治疗方案理解的 \underline{自回归(auto-regressive)} 分支,以及未来治疗后的多序列MRI影像(T1w, T1CE, Flair) 生成的 \underline{流式匹配(flow-matching)} 分支。ii) 在中间块进行治疗前后分割监督,平衡两项任务。
\item \textbf{项目成果}:该工作正在期刊投稿准备中。
\end{itemize}
}
\datedline{\textbf{基于生成式隐空间扩散模型的保真虚拟换装}}{\dateRange{2023.07}{2024.06}}
{\small
\begin{itemize}
  \item \textbf{项目简述}:与\underline{\emph{苏州象寄人工智能公司}}合作,针对目前生成式模型不保真多样性的问题,提出了一种全新用于虚拟换装的新型\underline{\emph{保真latent diffusion model}},称为 FLDM-VTON。 在基准 VITON-HD 和 Dress Code 数据集上的大量实验结果表明,我们的 FLDM-VTON 优于最先进的基线方法,能够生成具有忠实服装细节的逼真试穿图像。
\item \textbf{项目成果}:该工作被人工智能顶级CCF A类会议IJCAI 进行收录(计算机视觉方向中稿率仅为\textbf{8\%}),并获得\underline{Travel资助}前往国际大会上进行\underline{\emph{口头及海报展示}},该模型也参与一项相关发明专利(已公开),对应模型也被合作公司商用化使用。
\end{itemize}
}
\datedline{\textbf{物理遵循的音频驱动数字人视频生成}}{\dateRange{2025.06}{至今}}
{\small
\begin{itemize}
\item \textbf{项目简述}:与\underline{\emph{Soul AI lab}}合作,设计了一个物理遵循的音频驱动数字人视频生成框架,在\underline{视频编辑模型VACE}(Wan2.1 扩展版)基础上,额外利用X-Pose提取对应Pose特征,进而借助REPA思想设计\underline{物理状态离散扩散}为连续视频扩散提供额外监督先验,提升数字人生成质量;同时引入\underline{Qwen-2.5-Omni}为整体视频合成提供宏观指导,整体采用LoRA微调。
\item \textbf{项目成果}:该工作已投稿至CVPR 2026,并相关思想被合作公司商用整合。
\end{itemize}
}

\logosection{\faSuitcase}{发表论文}
\begin{justify}
{\small
\begin{enumerate}[left=0pt,itemsep=0pt]
\item \textbf{C. Wang}, S. Piao, Z. Huang, Q. Gao, J. Zhang, \underline{Y. Li}, and \underline{H. Shan}. ``\href{https://doi.org/10.1016/j.media.2023.103032}{Joint learning framework of cross-modal synthesis and diagnosis for Alzheimer’s disease by mining underlying shared modality information.}''  \emphi{\textbf{Med. Image Anal.}}, 91, 103032, 2024. \emphi{[\emph{3D MRI-to-PET synthesis|AD diagnosis|Joint learning framework}]}
    
\item \textbf{C. Wang}, Y. Lei, T. Chen, J. Zhang, \underline{Y. Li}, and \underline{H. Shan}. \href{https://ieeexplore.ieee.org/document/10412338}{``HOPE: Hybrid-granularity Ordinal Prototype Learning for Progression Prediction of Mild Cognitive Impairment.''} \emphi{\textbf{IEEE J. Biomed. Health Inform.}}, 28(11), 6429-6440,
  2024. \emphi{[\emph{AD ordinal progression|MCI prediction|Rank-based prototype learning}]}
  
  \item \textbf{C. Wang}, T. Chen, Z. Chen, Z. Huang, T. Jiang, Q. Wang, and \underline{H. Shan}. \href{https://www.ijcai.org/proceedings/2024/151}{``FLDM-VTON: Faithful Latent Diffusion Model for Virtual Try-on.''} \emphi{\textbf{IJCAI oral \& poster}}, 2024. \emphi{[\emph{Virtual Try-on|Latent diffusion}]}
  
  \item \textbf{C. Wang}$^{*}$, {S. Piao}$^{*}$, {J. Wang}$^{*}$, Z. Li, M. Cui, J. Zhao, 其他.
  ``AI-driven synthesis of multi-tracer PET from MRI enables
accurate Alzheimer’s disease diagnosis.'' 提交在2025. \emphi{[\emph{3D MRI-to-multi-tracer PET synthesis|DiT}]}
  
  \item \textbf{C. Wang}, S. Piao, Z. Chen, T. Chen, Z. Li, T. Zhang, 其他.
  ``E$^2$AD: Enhanced and Explainable Alzheimer’s Disease Detection Framework via Anatomy- and Relation-aware Cross-modal Knowledge Distillation.'' 提交在2025. \emphi{[\emph{Anatomical MoE|cross-modal distillation|AD detection}]}

\item \textbf{C. Wang}$^{*}$, L. Shen$^{*}$, J. Ye, Y. Jin, T. Yu, S. Liu, 其他.``Physically Grounded Diffusion Transformer for Audio-driven Avatar Generation.'' 提交在2025. \emphi{[\emph{Discrete diffusion|MLLM|DiT}]}

\item \textbf{C. Wang}, B. Zhen, L. Bao, Z. Peng, P. Woo, \underline{H. Shan}, and \underline{Y. Yuan}. ``Treatment-aware Brain Glioblastoma World Model''. \emphi{[world model|UMM]}

\item T. Chen, \textbf{C. Wang}, Z. Chen, Y. Lei, and \underline{H. Shan}. \href{https://ieeexplore.ieee.org/abstract/document/10587153/}{``HiDiff: Hybrid diffusion framework for medical image segmentation.''} \emphi{\textbf{IEEE Trans. Med. Imaging}}, 43(10), 3570-3583, 2024. \emphi{[\emph{Segmentation|Hybrid framework}]}

\item T. Chen, \textbf{C. Wang}, and \underline{H. Shan}. \href{https://link.springer.com/chapter/10.1007/978-3-031-43901-8_47}{``BerDiff: Conditional {Bernoulli} Diffusion Model for Medical Image Segmentation''.} \emphi{\textbf{MICCAI}}, 2023.
\emphi{[\emph{Segmentation|Diffusion model|Bernoulli}]}

\item T. Chen, \textbf{C. Wang},  Z. Chen, and \underline{H. Shan}. \href{https://www.arxiv.org/abs/2502.20784}{``Autoregressive Medical Image Segmentation via Next-Scale Mask Prediction.''} \emphi{\textbf{MICCAI}}, 2025. \emphi{[\emph{Segmentation|Autoregressive model|Next-scale}]}

\item Z. Li, \textbf{C. Wang}, Y. Li, and \underline{H. Shan}.
{``Imaging Biomarker Auto-Discovery Through Generative Artificial Intelligence.''}
准备在2026. 
\emphi{[\emph{AIGC|counterfactual image synthesis}]}

\item Z. Chen, T. Chen, \textbf{C. Wang}, Q. Gao, C. Niu, \underline{G. Wang}, and \underline{H. Shan}.\href{https://www.computer.org/csdl/proceedings-article/bibm/2024/10822519/23oopDIfoRy}{``Low-dose CT denoising with language-engaged dual-space alignment.''}  \emphi{\textbf{BIBM}}, 2024. \emphi{[\emph{Low-dose CT denoising|LLM-guided}]}

\item Z. Chen, T. Chen, \textbf{C. Wang}, Q. Gao, H. Xie, C. Niu, \underline{G. Wang}, and \underline{H. Shan}. \href{https://arxiv.org/abs/2507.06140}{``LangMamba: A Language-driven Mamba Framework for Low-dose CT Denoising with Vision-language Models.''} \emphi{\textbf{IEEE Trans. Radiat. Plasma Med. Sci.}} \emphi{[\emph{Low-dose CT denoising|LLM-guided|Mamba}]}

\item B. Cao, X. Yao, \textbf{C. Wang}, J. Ye, Y. Wei, and \underline{H. Shan}. {``Boosting Efficient Diffusion Transformer with
Dynamic Differential Linear Attention.''
} 提交在2025. 
\emphi{[\emph{MoE|linear attention|image generation}]}

\item 	J. Ye, G. Cong, \textbf{C. Wang}, X. Wen, Z. Li, B. Cao, \underline{H. Shan}.
{``Hierarchical Codec Diffusion for Video-to-Speech Generation.''}提交在2025.
\emphi{[Discrete diffusion|video-to-speech]}


\item T. Chen, Q. Niu, Z. An, \textbf{C. Wang},  Z. Chen, L. Du, 其他. ``Noninvasive Molecular Subtyping of Breast Cancer Using Multimodal Ultrasound Spatiotemporal Transformer.'' 提交至2025,\emphi{[\emph{Multimodal|DiT}]}

\item Y. Wei, C. Ma, J. Gao, \textbf{C. Wang}, S. Zhang, B. Gong, , 其他.
{``Bridging Brain and Semantics: A Hierarchical Framework for Semantically Enhanced fMRI-to-Video Reconstruction.''}  提交在2025. 
\emphi{[\emph{fMRI-to-Video generation|cross-modal semantic alignment|RAG}]}

\item C. Ma, Y. Ji, J. Ye, Z. Li, \textbf{C. Wang}, J. Ning, 其他.
\href{https://arxiv.org/abs/2505.19225}{``MedITok: A Unified Tokenizer for Medical Image Synthesis and Interpretation.''}  提交在2025. \emphi{[\emph{Unified Tokenizer|medical image synthesis|medical image interpretation}]}
\end{enumerate}
}
\end{justify}

\logosection{\faWrench}{其他内容}
\begin{itemize}
\item \textbf{审稿经历: }
\href{https://cvpr.thecvf.com/}{CVPR}, \href{https://2025.ijcai.org/}{IJCAI}, \href{https://asia.siggraph.org/2025/}{SIGGRAPH Asia}, \href{https://www.sciencedirect.com/journal/medical-image-analysis}{MedIA}, \href{https://ieeetmi.org/}{TMI}, \href{https://ieeexplore.ieee.org/xpl/RecentIssue.jsp?punumber=76}{TCSVT}, \href{https://conferences.miccai.org/2024/en}{MICCAI}, \href{https://ieeebibm.org/BIBM2024/}{BIBM}, \href{https://www.midl.io/}{MIDL} 等;
\item \textbf{社会工作: }  \underline{\emph{具有良好沟通及团队合作能力}},曾担任班级班长、及团支部书记,带领班级荣获 10 余项荣誉,并作为湖北省 30 名青年代表之一参加第六届“中俄‘长江-伏尔加河’青年论坛”并发言; 
\item \textbf{志愿服务: } \underline{\emph{性格温和、责任意识强}},累计志愿服务活动时数达 300 余小时,包括曾开展汉绣传承相关国家级“最具旅游影响力院校社会实践 TOP10”项目,及前往湖北省当阳市开展暑期支教项目。
\end{itemize}
